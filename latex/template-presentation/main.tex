%%%%%%%%%%%%%%%%%%%%%%%%%%%%%%%%%%%%%%%%%
% Beamer Presentation
% LaTeX Template
% Version 1.0 (10/11/12)
%
% This template has been downloaded from:
% http://www.LaTeXTemplates.com
%
% License:
% CC BY-NC-SA 3.0 (http://creativecommons.org/licenses/by-nc-sa/3.0/)
%
%%%%%%%%%%%%%%%%%%%%%%%%%%%%%%%%%%%%%%%%%
%\documentclass[xcolor = {usenames,dvipsnames,table}, handout]{beamer}

\documentclass[xcolor = {usenames,dvipsnames,table},aspectratio=169]{beamer}

%----------------------------------------------------------------------------------------
%	PACKAGES AND THEMES
%----------------------------------------------------------------------------------------

\usepackage{ulem}
\usepackage{breqn}
%\usepackage{subcaption}
%\captionsetup{compatibility=false}
%\usepackage{bbding}
\usepackage{ wasysym }
%\usepackage{pdfpages}
\usepackage{multirow}
\usepackage{caption}
\usepackage{algorithm}
\usepackage{algpseudocode}   %[noend]
\usepackage{epstopdf}
\usepackage{amsmath, amssymb}
\usepackage{tikz}                   
\usetikzlibrary{shadows}
\usepackage{colortbl}
\usepackage{adjustbox}
\usepackage[numbers]{natbib}
\usepackage{nicefrac}
\usepackage{graphicx} % Allows including images
\usepackage{booktabs} % Allows the use of \toprule, \midrule and \b	ottomrule in tables
\usepackage{xspace}
\usepackage{setspace}

%--- Table column commands with specified widths:
\usepackage{tabularx}
% L		left-aligned
% C		centered
% R		right-aligned
\newcolumntype{L}[1]%
{>{\raggedright\let\newline\\\arraybackslash\hspace{0pt}}m{#1}}
\newcolumntype{C}[1]%
{>{\centering\let\newline\\\arraybackslash\hspace{0pt}}m{#1}}
\newcolumntype{R}[1]%
{>{\raggedleft\let\newline\\\arraybackslash\hspace{0pt}}m{#1}}


% User commands
\newcommand{\citeauth}[1]{\citeauthor{#1} (\citeyear{#1})\xspace}
\newcommand{\rtb}{$\blacktriangleright$\xspace}
\newcommand*{\cmark}{\color{Green}$\checkmark$}%
\newcommand*{\xmark}{\color{Red}\large\textbf{\textit{$x$}}}%
\newcommand{\degc}{$^\circ$C\xspace}
\newcommand{\onm}{O\&M\xspace}
\newcommand{\scotu}{s-CO$_2$\xspace}
\newcommand{\st}{\text{subject to: }}
\newcommand{\zapspace}{\topsep=0pt\partopsep=0pt\itemsep=0pt\parskip=0pt}
%macro for specifying white vertical rule in tables. Replace | with !{\wl}
\newcommand{\wl}{\color{white}\vrule width 1.5pt}
%checkmarks for table
\newcommand{\bchk}{\Large$\checkmark$}
\newcommand{\gchk}{\textcolor{green!70!black}{\bchk}}


\mode<presentation> {

  \usetheme{CambridgeUS}
  \usecolortheme{seahorse}

  \setbeamertemplate{itemize item}{$\blacktriangleright$}
  \setbeamertemplate{itemize subitem}{--}
  \setbeamertemplate{itemize subsubitem}{$\bullet$}

  \newcommand{\smallitems}{
    \setbeamertemplate{itemize/enumerate body begin}{\small}
    \setbeamertemplate{itemize/enumerate subbody begin}{\footnotesize}
  }

  \newenvironment<>{postit}[1]%
    {\begin{block}{\vspace{-3ex}}#1}%
    {\end{block}} 

  \setbeamertemplate{navigation symbols}{} % To remove the navigation symbols from the bottom of all slides uncomment this line
}

\setbeamertemplate{blocks}[rounded][shadow=true]


\graphicspath{{./figures/}}

%--- definitions
%\setbeamercolor{block title}{use=structure,fg=white,bg=YellowOrange!80!Gray}
%\setbeamercolor{block body}{use=structure,fg=black,bg=YellowOrange!25!white}

%================================================================
\definecolor{uwred}{HTML}{C5050C}
\definecolor{uwreddark}{HTML}{9b0000}
\definecolor{uwpagebg}{HTML}{f7f7f7}
\definecolor{uwgrayblue}{HTML}{dadfe1}
\definecolor{uwgraydarker}{HTML}{646569}
\definecolor{uwgraydarkest}{HTML}{282728}
\definecolor{uwbodyfont}{HTML}{494949}

\setbeamercolor{palette primary}{bg=uwreddark,fg=white}
\setbeamercolor{palette secondary}{bg=uwgrayblue,fg=uwgraydarker}
\setbeamercolor{palette tertiary}{bg=uwred,fg=white}
\setbeamercolor{palette quaternary}{bg=uwgraydarker,fg=white}
\setbeamercolor{structure}{fg=uwreddark}
\setbeamercolor{section in toc}{fg=uwgraydarkest} % TOC sections
\setbeamercolor{normal text}{fg=uwbodyfont, bg=uwpagebg}
\setbeamercolor{frametitle}{fg=uwgraydarker, bg=uwgrayblue}
\setbeamercolor{block body}{fg=uwgraydarker, bg=white}
\setbeamercolor{block title}{fg=uwred, bg=uwgrayblue}


% -----------------------------------------------------------------
\newcommand{\titlecolor}{\usebeamercolor[fg]{title in head}}
\newcommand{\sectioncolor}{\color{structure}}
%\rowcolors{1}{author in head/foot.bg}{date in head/foot.bg}
\rowcolors{1}{uwgrayblue!60!white}{uwgrayblue}
\newcommand{\trhead}{\rowcolor{uwgraydarker}} 
\newcommand{\thf}{\bf\color{white}}

\renewcommand{\sectionpage}{
	\frame[noframenumbering]{\titlecolor \huge \insertsection}
}
\renewcommand{\subsectionpage}{
	\frame[noframenumbering]{\titlecolor \huge \insertsection \\ \vspace{.25in}\sectioncolor \Large \insertsubsection}
}
\usepackage{appendixnumberbeamer}


%----------------------------------------------------------------------------------------
%	TITLE PAGE
%----------------------------------------------------------------------------------------
\makeatletter
\let\inserttitle\@title
\makeatother

% The short title [] appears at the bottom of every slide, the full title is only on the title page
\title[XYZ Seminar]{An Interesting Thing I Did} 
\subtitle{On a Nice Day at the Terrace}

\author[Wagner]{Mike Wagner, PhD}
\date{\today} %{June 6\textsuperscript{th}, 2019}
\institute[UW-Madison]{ %
University of Wisconsin-Madison
}
%----------------------------------------------------------------------------------------
%----------------------------------------------------------------------------------------

\begin{document}

{
\setbeamertemplate{footline}{}
\setbeamertemplate{headline}{}
\usebackgroundtemplate{\includegraphics[width=\paperwidth,height=\paperheight]{title_slide}} 
\begin{frame}
  %\titlepage % Print the title page as the first slide
  \setstretch{1.3}
	\centering
  \vspace{0.15\textheight}
  \begingroup
  \color{white} \Huge
  \inserttitle \par
  \LARGE
  \vspace*{.5ex}\insertsubtitle 
  \endgroup

	%\includegraphics[width=0.8\textwidth]{CSM-logo}
	\vspace{0.15\textheight} %\hspace{0.6\linewidth}
	\begin{minipage}{0.8\linewidth}
  % \raggedright 
  \centering \large 
  \color{white} 
    \insertinstitute \par 

		\insertauthor \par
    
    \insertdate \par 

	\end{minipage}
\end{frame}
}

%{
%\usebackgroundtemplate{\includegraphics[width=\paperwidth,height=\paperheight,draft=false]{images/title_slide}} 
%\begin{frame}
%\titlepage % Print the title page as the first slide
%\end{frame}
%}
%
% \usebackgroundtemplate{\includegraphics[width=\paperwidth, height=\paperheight,draft=false]{standard_slide}}
%
%\setbeamertemplate{frametitle}{\vskip6pt\bfseries\insertframetitle}



%----------------------------------------------------------------------------------------
%	PRESENTATION SLIDES
%----------------------------------------------------------------------------------------


\newenvironment{blocki}[1]{%
  \setbeamercolor{block body}{use=structure,fg=black,bg=uwgrayblue}
  \setbeamercolor{block title}{use=structure,fg=white,bg=uwreddark}
  \begin{block}{#1}}{\end{block}}

% \newenvironment{blockii}[1]{%
%   \setbeamercolor{block body}{use=structure,fg=black,bg=nrelslate!40}
%   \setbeamercolor{block title}{use=structure,fg=white,bg=nrelslate}
%   \begin{block}{#1}}{\end{block}}

% \newenvironment{blockiii}[1]{%
%   \setbeamercolor{block body}{use=structure,fg=black,bg=nrelorange!40}
%   \setbeamercolor{block title}{use=structure,fg=black,bg=nrelorange}
%   \begin{block}{#1}}{\end{block}}




\captionsetup{justification=centering}

% Separate sections with \sectionpage

\begin{frame}
    \frametitle{Contents}

    \tableofcontents

\end{frame}

% ----------------------------------------------------------------
% ----------------------------------------------------------------
\section{Old work}

% ----------------------------------------------------------------
\begin{frame}
    \frametitle{Other people have done work}

    \begin{columns}
        \column{0.5\linewidth}
            \centering 
            \includegraphics[width=\linewidth]{title_slide.png}
        
        \column{0.4\linewidth}

            \begin{enumerate}
                \item First task
                \item Second task
                \item Third task
                \item Fourth
            \end{enumerate}

    \end{columns}

\end{frame}
% ----------------------------------------------------------------


% ----------------------------------------------------------------
% ----------------------------------------------------------------
\section{New work} \sectionpage

% ----------------------------------------------------------------
\begin{frame}
    \frametitle{I have also done some work}

    \begin{columns}

        % Choose how to vertically align column content (t, c, b)
        \column[c]{0.6\linewidth}

            \begin{blocki}
                {Here is an important idea or proof}
                Plank's law is cited \citep{planck1914theory} and reproduced in Eq.~\ref{eq:1}:

                \begin{equation}
                    \int_0^\infty E_{b,\lambda} d\lambda = \frac{C_1}{\lambda^5 \left[ \exp\left( \frac{C_2}{\lambda \cdot T} \right) - 1 \right]}  
                    \label{eq:1}
                \end{equation}


            \end{blocki}

        \column[c]{0.3\linewidth}

            \begin{itemize}
                \item Here's a list
                \item With a few items
                \item To guide the talk
            \end{itemize}


    \end{columns}

\end{frame}
% ----------------------------------------------------------------


% ----------------------------------------------------------------
\begin{frame}[fragile]
    \frametitle{A table is necessary}

    \begin{columns}
        \column{0.5\linewidth}
        
            \centering 
        
            \begin{table}
                \caption{This is a sample table. Note the format specifiers on the header row!}
                \label{tab:1}
                % Separate columns with a white line using macro '!{\wl}' instead of '|'
                \begin{tabular}{L{1in} !{\wl} c  c}
                    \trhead \thf Description & \thf Units & \thf Value \\
                    Flux & $\nicefrac{W}{m^2}$ & 100.0 \\
                    Cost & \$ & 13,000,000 \\
                    Specific heat of the third to last fluid which has a very long wrapped label &
                        $\frac{kJ}{kg \cdot K}$ & 985 \\
                \end{tabular}
            \end{table}

        \column{0.4\linewidth}

            Here is the code for the \texttt{tabular} environment:

            \tiny

            \begin{verbatim}
\begin{tabular}{L{1in} !{\wl} c  c}
    \trhead \thf Description & \thf Units & \thf Value \\
    Power & $W$ & 1000 \\
    ...
\end{tabular}
            \end{verbatim}
        
    \end{columns}

\end{frame}
% ----------------------------------------------------------------


% Uncomment the following if using a bibliography
%%------------------------------------------------
\begin{frame}[allowframebreaks]
  {References}
\footnotesize
\bibliographystyle{unsrtnat} %abbrvnat}
\bibliography{library}
\end{frame}
%%------------------------------------------------


\end{document} 